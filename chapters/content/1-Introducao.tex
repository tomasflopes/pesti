\chapter{Introdução} 	
\label{sec:1-Introducao}

Este primeiro capítulo estabelece as bases necessárias para a compreensão do trabalho desenvolvido. 
Primeiramente, é exposta a motivação do trabalho e o seu enquadramento no contexto da Blip. 
De seguida, são referidos os principais objetivos identificados, a abordagem adotada, os contributos 
da realização do projeto e uma apresentação da estrutura que o documento segue.

\section{Enquadramento}

Este documento é o resultado do projeto de estágio desenvolvido na Blip durante o sexto semestre 
da \ac{LEI} do \ac{ISEP} no âmbito da \ac{UC} de \ac{PESTI} no ano letivo de 2023/2024. A Blip é 
uma empresa tecnológica que pertence ao grupo Flutter que desenvolve soluções de \textit{software} 
para apostas desportivas online, sendo, neste momento, o maior grupo de apostas desportivas online 
a nível global \cite{blip}.

O projeto apresentado está inserido na ordem de trabalhos na equipa de \textit{Application Engineering},
\textit{Watchmen}, responsável por lidar com desafios técnicos de entrega dos produtos, de forma
a libertar as equipas de desenvolvimento dos desafios ligados a infraestrutura.

O sistema que é usado pela Blip deve estar preparado para lidar com um grande volume de dados em 
tempo real e, por consequência, é bastante complexo. Ao longo deste relatório vai ser analisado 
o sistema que lida com este dados que são recebidos de várias fontes e servem de base para a criação 
de eventos e informações dos mesmos. Este sistema é baseado em \textit{Apache Storm} e 
\textit{Apache Kafka}, que são tecnologias que permitem processar e armazenar os dados em tempo 
real de forma eficiente tentando evitar ao máximo que haja constrangimentos no sistema.

\section{Descrição do Problema}

O \gls{cluster} que lida com os dados de catálogo é composto por várias máquinas que têm
como função receber e transformar um grande volume de dados de vários eventos que são recebidos de 
diversas fontes. Para controlar este \gls{cluster} é usada a tecnologia \textit{Apache Storm}
para fazer o processamento destes dados.

Com o crescimento da empresa, o volume de dados que é processado tem vindo a aumentar e, além disso,
nos últimos meses o grupo Flutter adquiriu uma nova empresa para a divisão de \ac{UKI} o que fez 
com que o volume de dados que o \gls{cluster} tem que ser capaz de processar aumentasse 
substancialmente.

Desta forma, o problema que se apresenta é que este \gls{cluster} não tem capacidade para lidar
com o volume de dados para o qual está a ser sujeito e, por isso, é necessário fazer algumas
otimizações por forma a não haver constrangimentos no sistema. A abordagem mais simplista poderia 
passar por aumentar apenas a capacidade de processamento do \gls{cluster}, mas isso seria uma 
solução bastante dispendiosa e, provavelmente, traria problemas de escalabilidade no futuro.

Assim, é necessário fazer uma análise da utilização dos recursos do sistema e perceber onde estão 
os principais problemas de performance e escalabilidade de forma a tentar encontrar soluções que 
possam ser implementadas para resolve-los. Numa primeira fase, é necessário perceber como o sistema 
está implementado na infraestrutura da empresa e fazer uma análise aos recursos usados por cada 
máquina que compõe o \gls{cluster} de forma a perceber onde estão os principais problemas. De 
seguida, será realizada uma atualização da versão do \textit{Apache Storm} que é usada no sistema 
de forma a tentar tirar partido das funcionalidades e melhorias de desempenho que foram 
implementadas nas versões mais recentes.

\subsection{Objetivos}

Os objetivos identificados têm como finalidade a resolução do problema apresentado e a melhoria do
desempenho do sistema. Desta forma, os objetivos a atingir são os seguintes:

\begin{itemize}
  \item Identificar desafios de escalabilidade e performance
  \item Familiarização com a ferramenta \textit{Apache Storm }
  \item Testar e implementar Apps Storm \textit{(Java Tech Stack)}
  \item Identificar e propor soluções de integração com \textit{Apache Kafka}
  \item Ajudar a definir o planeamento de atualização e \textit{rollback}
\end{itemize}

\subsection{Abordagem}

Primeiramente, será necessária fazer uma análise às métricas de sistema de forma a compreender
o uso de recursos do \gls{cluster}. Para isso, é usada a ferramenta Grafana que permite
monitorizar um conjunto de métricas de sistema e de aplicação, de forma a identificar os principais
problemas de escalabilidade. Além disso, a monitorização destas métricas vai ser crucial para
garantir que as alterações efetuadas no projeto não afetam o bom funcionamento do sistema.

De seguida, será elaborado um plano de migração dos \glspl{cluster} de forma a minimizar o 
impacto destas alterações nos sistemas que se encontram em produção para os clientes destes serviços.

No caso da Blip, os sistemas correm em vários ambientes e replicados em dois \textit{\glspl{dc}},
logo, o plano de migração terá em conta estas questões de forma a tirar partido desta infraestrutura
e minimizar a indisponibilidade e possibilidade de erros nos sistemas produtivos.

Para levar a cabo este trabalho será utilizada a abordagem \textit{agile} que é uma abordagem
iterativa e incremental que permite a adaptação a mudanças de forma rápida e eficiente. Esta abordagem
é bastante utilizada em projetos de desenvolvimento de \textit{software} e permite a entrega de valor
de forma rápida e eficiente.

\subsection{Contributos}

Este trabalho irá ser crucial para ser possível manter o funcionamento destes sistemas na
infraestrutura atual da empresa.

\todo{TODO: Elaborar}

\subsection{Planeamento do Trabalho}

O planeamento do trabalho concentra-se na organização e divisão do tempo útil entre as várias etapas
que serão concluídas de forma a atingir a solução final. A Tabela \ref{tab:plan}, apresenta a visão
geral do planeamento elaborado.

\begin{table}[H]
  \begin{center}
    \caption{Planeamento do Trabalho}
    \vspace{5mm}
    \label{tab:plan}
    \begin{tabular}{|c|c|c|}
      \hline
      \textbf{Etapa} & \textbf{Data Início} & \textbf{Duração} \\ \hline
      Familiarização com Apache Storm  & 19/02/2024 & 2 semanas \\ \hline
      Análise das otimizações de recursos & 04/03/2024 & 4 semanas \\ \hline
      Implementação das otimizações & 01/04/2024 & 6 semanas \\ \hline
      Upgrade Apache Storm & 13/05/2024 & 4 semanas \\ \hline
    \end{tabular}
  \end{center}
\end{table}

\section{Estrutura do Relatório}

O presente relatório apresenta cinco capítulos, sendo estes: \nameref{sec:1-Introducao},
\nameref{sec:2-EstadoArte}, \nameref{sec:3-Analise}, \nameref{sec:4-Implementacao} e
\nameref{sec:5-Conclusoes}.

O primeiro capítulo - \nameref{sec:1-Introducao} - faz uma breve contextualização do projeto de
forma a dar a conhecer a organização onde este foi realizado e uma descrição do problema que motivou
a solução apresentada. São também explicitados os objetivos a alcançar, a abordagem a seguir, os
contributos efetuados, o planeamento do trabalho adotado e a estrutura do documento. Esta secção é 
fundamental para que o leitor consiga acompanhar o processo de desenvolvimento do projeto.

O segundo capítulo - \nameref{sec:2-EstadoArte} - tem como objetivo realizar uma revisão da
literatura, com o intuito de aprofundar assim alguns conceitos científicos e tecnologias relevantes 
para contextualizar o leitor no domínio teórico do projeto.  

O terceiro capítulo - \nameref{sec:3-Analise} - tem como propósito fornecer uma descrição completa 
do desenvolvimento da solução e como o projeto funcionará na sua totalidade, abordando tanto conceitos
de domínio do problema como também os requisitos funcionais e não funcionais.

O quarto capítulo - \nameref{sec:4-Implementacao} - visa apresentar a solução desenvolvida e 
descrever detalhes de implementação, assim como explicações sobre as decisões tomadas durante o
desenvolvimento do projeto e possíveis alternativas.

O quinto, e último, capítulo - \nameref{sec:5-Conclusoes} - realiza uma síntese dos resultados 
alcançados com o desenvolvimento do projeto, limitações encontradas, bem como uma perspetiva de 
futuras melhorias e considerações finais sobre o trabalho realizado.

No final do documento são também disponibilizados alguns anexos e conteúdos bibliográficos que 
suportam o trabalho desenvolvido apresentado ao longo deste relatório.
