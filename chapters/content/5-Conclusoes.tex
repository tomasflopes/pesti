\chapter{Conclusões}
\label{sec:5-Conclusoes}

Este capítulo final, apresenta uma síntese dos pontos mais relevantes do trabalho desenvolvido. 
Primeiramente, é efetuada uma validação dos objetivos propostos inicialmente. De seguida, são 
analisadas as limitações e as possíveis melhorias futuras. Na última secção, é realizada uma 
apreciação crítica com o intuito de salientar pontos positivos e negativos enfrentados ao longo do 
desenvolvimento do projeto.

\section{Objetivos concretizados}

A Tabela \ref{tab:obj} ilustra o grau de realização de cada um dos objetivos propostos 
inicialmente. Como é possível verificar, quase todos os objetivos foram alcançados com sucesso.

A otimização do ambiente de produção não foi possível levar a cabo até ao final devido aos 
períodos de proteção de negócio devido a vários eventos de grande escala que ocorreram durante o
período de desenvolvimento do projeto como é o caso das finais da FA Cup, da Liga dos Campeões e
os Europeus de Futebol. Mesmo assim, foi possível realizar uma análise detalhada do ambiente de
produção e identificar que a sua otimização é possível sem riscos associados.

\begin{table}[H]
  \begin{center}
    \caption{Visão geral dos objetivos técnicos alcançados}
    \vspace{2mm}
    \label{tab:obj}
    \begin{tabular}{|l|c|}
      \hline
      \textbf{Objetivo} &
      \multicolumn{1}{l|}{\textbf{Grau de realização}} \\ \hline
      \begin{tabular}[c]{@{}l@{}}Otimização do ambiente de \ac{QA} \end{tabular} & 100\% \\ \hline
      \begin{tabular}[c]{@{}l@{}}Otimização do ambiente de performance \end{tabular} & 100\% \\ \hline
      \begin{tabular}[c]{@{}l@{}}Otimização do ambiente de desenvolvimento \end{tabular} & 100\% \\ \hline
      \begin{tabular}[c]{@{}l@{}}Otimização do ambiente de produção \end{tabular} & 50\% \\ \hline
    \end{tabular}
  \end{center}
\end{table}

\section{Limitações e trabalho futuro}

Como mencionado anteriormente, a maior limitação encontrada foi a impossibilidade de otimizar o
ambiente de produção devido a eventos de grande escala que ocorreram durante o período de
desenvolvimento do projeto. 

Posto isto, o trabalho futuro passa por otimizar o ambiente de produção e realizar uma análise 
detalhada ao impacto da otimização em todos os ambientes. Além disso, uma das possíveis otimizações 
futuras seria proceder à atualização do \textit{Apache Storm} que está a ser executado pelos 
\glspl{cluster}. Ao atualizar o \textit{software} seria possível tirar partido de mais otimizações 
de performance das novas versões, o que contribuiria ainda mais para os objetivos de otimização
de uso de recursos traçados inicialmente.

\section{Apreciação final}

A realização deste estágio curricular no âmbito de questões de infraestrutura revelou-se um desafio 
estimulante e uma oportunidade de crescimento significativo, tanto a nível profissional como 
pessoal. Ao longo do percurso meu académico na \ac{LEI}, a ênfase foi principalmente na área do 
desenvolvimento e desenho de \textit{software}, o que tornou a imersão numa área mais específica 
uma experiência bastante enriquecedora.

Desde o início do estágio, fui acolhido da melhor forma pela equipa, que demonstrou um elevado 
nível de profissionalismo, o acompanhamento contínuo foi essencial para que a minha integração e 
aprendizagem decorressem de uma maneira simples. A complexidade técnica e contextual do ponto de 
vista do negócio do projeto apresentou desafios consideráveis, mas proporcionou uma experiência 
prática, onde pude explorar novas áreas e enfrentar novos desafios de dimensões impossíveis noutro 
contexto.

Durante este percurso enfrentei algumas dificuldades, naturais num projeto desta envergadura, que 
foram superadas com apoio da equipa. A resolução desses obstáculos foi crucial para o desenvolvimento
da solução apresentada. Além disso, O ambiente colaborativo e a troca de ideias constantes foram 
fundamentais para o sucesso alcançado.

Em resumo, considero que este estágio foi extremamente proveitoso e uma mais-valia tanto a nível
técnico como interpessoal. As competências adquiridas e as novas perspetivas ganhas serão, sem 
dúvida, determinantes para o meu futuro profissional. Posto isto, estou satisfeito com os resultados 
obtidos e grato pela oportunidade de participar num projeto tão desafiador.

