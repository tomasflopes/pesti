\chapter{Conclusões}
\label{sec:5-Conclusoes}

Este capítulo final, apresenta uma síntese dos pontos mais relevantes do trabalho desenvolvido. 
Primeiramente, é efetuada uma validação dos objetivos propostos inicialmente. De seguida, são 
analisadas as limitações e as possíveis melhorias futuras. Na última secção, é realizada uma 
apreciação crítica com o intuito de salientar pontos positivos e negativos enfrentados ao longo do 
desenvolvimento do projeto.

\section{Objetivos concretizados}

% Conforme é possível verificar nos capítulos \nameref{sec:2-EstadoArte}, \nameref{sec:3-Analise} 
% e \nameref{sec:4-Implementacao} todos os objetivos traçados na subsecção \nameref{sec:1-obj} 
% foram atingidos na sua totalidade. A Tabela \ref{tab:obj} ilustra o grau de realização de cada um
% dos objetivos inicialmente propostos.

\todo[inline]{TODO}

\begin{table}[H]
  \begin{center}
    \caption{Visão geral dos objetivos técnicos alcançados}
    \vspace{2mm}
    \label{tab:obj}
    \begin{tabular}{|l|c|}
      \hline
      \textbf{Objetivo} &
      \multicolumn{1}{l|}{\textbf{Grau de realização}} \\ \hline
      \begin{tabular}[c]{@{}l@{}}xxx\end{tabular} &
      xx\% \\ \hline
    \end{tabular}
  \end{center}
\end{table}

\section{Limitações e trabalho futuro}

\todo[inline]{TODO}

\section{Apreciação final}

\todo[inline]{TODO}

