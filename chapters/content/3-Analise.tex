\chapter{Análise e Desenho da Solução}
\label{sec:3-Analise}

Depois de contextualizados os temas e assuntos relevantes, este capítulo concentra-se na análise 
e elucidação do problema que sustenta o presente relatório e na apresentação do desenho da 
solução criada.

\section{Domínio do Problema}


\section{Engenharia de Requisitos}

A Engenharia de Requisitos é uma área muito relevanto no desenvolvimento de \textit{software}, pois 
sustenta o sucesso dos projetos. Representa o processo de obtenção de requisitos através de uma análise 
do problema e pressupõe a definição das necessidades do cliente na procura de uma solução clara 
que valide a proposta. Seguindo um processo estruturado e adotando as melhores práticas, 
promovemos uma melhor comunicação entre as várias partes interessadas.

Considerando os aspetos mencionados, nesta secção serão apresentados todos os 
requisitos do sistema identificados e requisitados no início do projeto de maneira a garantir a 
qualidade da solução desenvolvida. Estes requisitos podem ser categorizados em funcionais - 
funcionalidades distintas e essenciais que o sistema deve realizar, e não funcionais - 
restrições impostas para que o sistema realize os requisitos funcionais corretamente.

\subsection{Requisitos Não Funcionais}

Os requisitos não funcionais não se concentram no que um sistema de software faz, mas sim em como 
ele funciona. São essenciais para a qualidade geral, desempenho e usabilidade do software e 
consideram fatores como o desempenho, a segurança, a confiabilidade e a usabilidade.

Os requisitos não funcionais apresentados em seguida, guiam-se pelo modelo FURPS+, um padrão de 
classificação qualitativa das características de um \textit{software}
(\textbf{F}unctionality, \textbf{U}sability, \textbf{R}eliability, \textbf{P}erformance, \textbf{S}upportability), 
para uma melhor experiência do utilizador. O "\textbf{+}" refere-se a métodos de classificação 
diferentes, como por exemplo, restrições de design, implementação, interface ou físicos.

\vspace{5mm}

\textbf{Funcionalidade}
\begin{itemize}
  \item Encontram-se especificados na subsecção \nameref{sec:3-rf}.
\end{itemize}

\textbf{Usabilidade}
\begin{itemize}
  \item xxx
\end{itemize}

\textbf{Confiabilidade}
\begin{itemize}
  \item A aplicação deve ser capaz de recuperar de falhas sem perda de dados.
  \item A aplicação deve ser capaz de recuperar de falhas minimizando o tempo de inatividade.
  \item A aplicação deve ser capaz de recuperar de falhas sem intervenção manual.
\end{itemize}

\textbf{Desempenho}
\begin{itemize}
  \item O sistema deve ser capaz de processar transações em dia de pico sem falhas.
  \item O sistema deve ser capaz de processar transações em dia de pico minimizando o tempo de resposta.
\end{itemize}

\textbf{Suportabilidade}
\begin{itemize}
  \item xxx
\end{itemize}

\textbf{Restrições de Design}
\begin{itemize}
  \item xxx
\end{itemize}

\subsection{Requisitos Funcionais}
\label{sec:3-rf}

Os requisitos funcionais especificam as unidades e recursos funcionais de um sistema de software, 
concentrando-se nas funções que ele deve realizar. Eles descrevem as funcionalidades, comportamentos
e operações específicas que os utilizadores devem conseguir executar podendo variar entre ações básicas, 
como entrada e saída de dados, a algoritmos complexos e processos de negócios.

De forma a facilitar a compreensão por parte do leitor os requisitos funcionais,
encontram-se descritos, na Tabela \ref{tab:reqfun} e na Figura \ref{dcu}, na forma de \textit{User Stories}, 
seguindo a estrutura apresentada no artigo “(User) Stories for Analytics Projects – Part 1” \cite{us}.

\begin{table}[H]
  \begin{center}
    \caption{Requisitos Funcionais}
    \vspace{5mm}
    \label{tab:reqfun}
    \begin{tabular}{|c|l|}
      \hline
      ID & User Story                                                                                                           \\ \hline
      1  & \begin{tabular}[c]{@{}l@{}}Como XXX, quero XXX.\end{tabular} \\ \hline
      2  & \begin{tabular}[c]{@{}l@{}}Como XXX, quero XXX.\end{tabular} \\ \hline
    \end{tabular}
  \end{center}
\end{table}

\begin{figure}[H]
%  \centerline{\includegraphics[scale=0.55]{media/plantuml.png}}
  \caption{Diagrama de Casos de Uso}
  \label{dcu}
\end{figure}

\section{Desenho da Solução}

Através da análise do problema definido, em conjunto com os requisitos funcionais e não funcionais 
delineados anteriormente, o principal objetivo desta secção é documentar cada fase distinta que 
compõe o desenho da solução idealizada.

