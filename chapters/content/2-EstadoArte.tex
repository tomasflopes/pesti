\chapter{Estado da Arte}
\label{sec:2-EstadoArte}

O presente capítulo visa a contextualização teórica do trabalho realizado. Primeiramente, são 
abordados conceitos relacionados com o projeto. De seguida, é realizado um levantamento das 
tecnologias existentes no âmbito do projeto. Por fim, são expostas algumas soluções semelhantes 
já existentes no mercado, proporcionando assim uma visão ampla do contexto em que o trabalho 
desenvolvido se insere.

\section{Sistemas Distribuídos}

Os sistemas distribuídos são sistemas de \textit{software} que interagem entre si através de uma rede
de computadores. Estes sistemas são compostos por um conjunto de computadores autónomos que
apresentam uma visão única e coerente para os utilizadores. A comunicação entre os computadores
é realizada através de mensagens, permitindo a partilha de recursos e a execução de tarefas em
paralelo \cite{verissimo2001distributed}.

\todo{TODO: Elaborar, + referências}

\subsection{Arquitetura Cliente-Servidor}

\todo[inline]{TODO: Falar sobre a arquitetura cliente-servidor, conceitos e aplicabilidade.}

\subsection{Arquitetura Streaming}

Os sistemas de \textit{streaming} distribuídos são cruciais para o processamento e a análise de dados 
em tempo real. O desempenho destes sistemas depende, em grande parte, da distribuição efetiva da 
carga de trabalho entre as máquinas \cite{stream2020}. Para garantir um serviço de \textit{streaming} 
eficaz e confiável, é essencial estabelecer um ambiente distribuído para o fluxo contínuo de 
meios de comunicação \cite{stream2014}. Além disso, a computação de \textit{streams} surgiu como 
uma tecnologia líder para analisar e gerir dados de fluxo massivo, tornando-se um modelo popular
para a análise de fluxo de dados em tempo real \cite{stream2018} \cite{stream2018b}.

Além disso, o Storm, um sistema de \textit{streaming} distribuído, foi reconhecido como um sistema de 
processamento de \textit{streams} de dados distribuídos tolerante a falhas em tempo real, enfatizando
a importância da tolerância a falhas no processamento de \textit{streams} \cite{stormattwitter}.
Num sistema de processamento distribuído de \textit{streams}, estes dados são processados em tempo real 
por um conjunto de operadores distribuídos por um \textit{cluster} de servidores, o que demonstra a 
natureza distribuída dos sistemas de processamento de fluxos de dados.

\section{Estratégias de implantação}

\todo[inline]{Introdução sobre a importância de estratégias de implantação em sistemas distribuídos.}

\subsection{Canary Deployment}

\todo[inline]{Falar sobre canary deployment, conceitos, vantagens, desvantagens, etc.}

\subsection{Rolling Update}

\todo[inline]{Falar sobre rolling update, conceitos, vantagens, desvantagens, etc.}

\subsection{Blue-Green Deployment}

\todo[inline]{Falar sobre blue-green deployment, conceitos, vantagens, desvantagens, etc. (este é o que é usado
na Blip)}

\section{Monitorização de Sistemas}

\todo[inline]{Falar sobre monitorização de sistemas, conceitos, ferramentas. Grafana, Splunk, etc.}

\subsection{Métricas}

\todo[inline]{Falar sobre monitorização de métricas, falar sobre métricas de sistema e aplicação}

\subsection{Registo de eventos}

\todo[inline]{Falar sobre monitorização de logs, alertas}
