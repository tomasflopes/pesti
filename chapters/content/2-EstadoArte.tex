\chapter{Estado da Arte}
\label{sec:2-EstadoArte}

O presente capítulo visa a contextualização teórica do trabalho realizado. Primeiramente, são 
abordados conceitos relacionados com o projeto. De seguida, é realizado um levantamento das 
tecnologias existentes no âmbito do projeto. Por fim, são expostas algumas soluções semelhantes 
já existentes no mercado, proporcionando assim uma visão ampla do contexto em que o trabalho 
desenvolvido se insere.

\section{Sistemas Distribuídos}

Os sistemas distribuídos são sistemas de \textit{software} que interagem entre si através de uma rede
de computadores. Estes sistemas são compostos por um conjunto de computadores autónomos que
apresentam uma visão única e coerente para os utilizadores. A comunicação entre os computadores
é realizada através de mensagens, permitindo a partilha de recursos e a execução de tarefas em
paralelo \cite{verissimo2001distributed}.

\subsection{Arquitetura Cliente-Servidor}

Falar sobre a arquitetura cliente-servidor, conceitos e aplicabilidade.

\subsection{Arquitetura Streaming}

Falar sobre a arquitetura de streaming, Kafka, RabbitMQ, etc.

\section{Estratégias de implantação}

Introdução sobre a importância de estratégias de implantação em sistemas distribuídos. Falar sobre
as estratégias mais comuns, como blue-green deployment, canary deployment, rolling update, etc.

\subsection{Blue-Green Deployment}

Falar sobre blue-green deployment, conceitos, vantagens, desvantagens, etc. (este é o que é usado
na Blip)

\subsection{Canary Deployment}

Falar sobre canary deployment, conceitos, vantagens, desvantagens, etc.

\subsection{Rolling Update}

Falar sobre rolling update, conceitos, vantagens, desvantagens, etc.

\section{Monitorização de Sistemas}

Falar sobre monitorização de sistemas, conceitos, ferramentas. Grafana, Splunk, etc.

\subsection{Métricas}

\subsection{Registo de eventos}

