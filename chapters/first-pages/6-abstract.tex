\titleformat{\chapter}[display]
{\normalfont\bfseries}{}{0pt}{\Huge}

\chapter*{Abstract}

Given the context in which Blip operates, the company needs a system that allows it to process
large volumes of data in real time, in this case catalog data that is received from various sources 
and  sources and serve as the basis for creating events and their information. In this way it was 
developed a distributed system based on a streaming architecture in order to process and store 
data efficiently. The aim of this project is to make changes to the existing system in order to 
improve its efficiency and scalability, to ensure that the system is capable of is capable of 
handling an increased workload.

This document presents the development of an internship project carried out at Blip as part of the 
curricular unit of \ac{PESTI} of \ac{LEI} at \ac{ISEP}. 

The results obtained show that the changes made to the system made it possible to improve its
efficiency and scalability, ensuring that the system will be able to cope with the increase in
to which it will be subject in the future.

\textbf{\\Keywords (Themes):} 

Data processing, Distributed systems, Real-time systems

\textbf{\\Keywords (Technologies):} 

Apache Storm, Apache Kafka, Java, Nimbus, Zookeeper
