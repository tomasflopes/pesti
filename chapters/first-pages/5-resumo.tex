\titleformat{\chapter}[display]
{\normalfont\bfseries}{}{0pt}{\Huge}

\chapter*{Resumo}

Dado o contexto em que a Blip se insere, a empresa necessita de um sistema que permita processar
grandes volumes de dados em tempo real, neste caso, dados de catálogo que são recebidos de várias fontes
e servem de base para a criação de eventos e informações dos mesmos. Desta forma, foi desenvolvido 
um sistema distribuído baseado numa arquitetura de streaming de forma a processar e armazenar os 
dados de forma eficiente. O objetivo deste projeto é fazer alterações no sistema existente de forma 
a melhorar a sua eficiência e escalabilidade, por forma a garantir que o sistema é capaz de lidar 
com um aumento de carga de  trabalho.

Este documento apresenta o desenvolvimento de um projeto de estágio realizado na Blip no âmbito da 
unidade curricular de \ac{PESTI} da \ac{LEI} no \ac{ISEP}. 

Os resultados obtidos ...

\todo{TODO: Elaborar}


\textbf{\\Palavras-chave (Tema): } 

Processamento de dados, Sistemas distribuídos, Sistemas de tempo real

\textbf{\\Palavras-chave (Tecnologias):}

Apache Storm, Apache Kafka, Java, Nimbus, Zookeeper


