\chapter{Introdução} 	
\label{sec:1-Introducao} % For referencing the chapter elsewhere, usage \ref{Chapter1}

Este primeiro capítulo estabelece as bases necessárias para uma compreensão sólida do trabalho desenvolvido.
Primeiramente é exposta a motivação do trabalho e o seu enquadramento no contexto da Blip. De seguida,
são referidos os principais objetivos identificados, a abordagem adotada, os contributos da realização 
do projeto e uma apresentação da estrutura do documento.

\section{Enquadramento}

Este documento é o resultado do estágio desenvolvido na Blip durante o sexto semestre da Licenciatura 
em Engenharia Informática do ISEP no âmbito da Unidade Curricular de Projeto / Estágio (PESTI) 
no ano letivo de 2023/2024. A Blip é uma empresa ...

\section{Descrição do Problema}



\subsection{Objetivos}
\label{sec:1-obj}

\begin{itemize}
  \item Identificar desafios de escalabilidade e performance
  \item Familiarização com a ferramenta Apache Storm 
  \item Testar e implementar Apps Storm (Java Tech Stack) 
  \item Identificar e propor soluções de integração com Apache Kafka 
  \item Ajudar a definir o planeamento de atualização e rollback 
\end{itemize}

\subsection{Abordagem}



\subsection{Contributos}



\subsection{Planeamento do Trabalho}

O planeamento do trabalho concentra-se na organização e divisão do tempo útil entre as 
várias etapas que devem ser concluídas para de forma a atingir a solução final. A Tabela 1.1, 
apresenta a vista geral do planeamento elaborado.

\begin{table}[H]
  \begin{center}
    \caption{Planeamento do Trabalho}
    \vspace{5mm}
    \label{tab:plan}
    \begin{tabular}{|c|c|c|}
      \hline
      \textbf{Etapa} & \textbf{Data Início} & \textbf{Duração} \\ \hline
      Familiarização com Apache Storm  & xx/xx/2024 & 2 semanas \\ \hline
      Análise das otimizações de recursos & xx/xx/2024 & 4 semanas \\ \hline
      Implementação das otimizações & xx/xx/2024 & 6 semanas \\ \hline
      Upgrade Apache Storm & xx/xx/2024 & 4 semanas \\ \hline
    \end{tabular}
  \end{center}
\end{table}

\section{Estrutura do Relatório}

O presente relatório apresenta cinco capítulos, sendo estes: \nameref{sec:1-Introducao}, \nameref{sec:2-EstadoArte}, 
\nameref{sec:3-Analise}, \nameref{sec:4-Implementacao} e \nameref{sec:5-Conclusoes}.

O primeiro capítulo – \nameref{sec:1-Introducao} – faz uma breve contextualização do projeto de forma a dar a
conhecer a organização onde este foi realizado e uma descrição do problema que motivou a solução apresentada. 
São também explicitados os objetivos a alcançar, a abordagem a seguir, os contributos esperados, o planeamento do 
trabalho adotado e a estrutura do documento. Esta secção é fundamental para que o leitor consiga acompanhar o
processo de desenvolvimento do projeto.

O segundo capítulo – \nameref{sec:2-EstadoArte} – visa realizar uma revisão de literatura, com o intuito de 
aprofundar assim alguns conceitos científicos e tecnologias relevantes para contextualizar o leitor no domínio 
teórico e prático do projeto. São também apresentados trabalhos da área de negócio relacionados com o projeto desenvolvido.

O terceiro capítulo – \nameref{sec:3-Analise} – tem como propósito fornecer uma
descrição completa do desenvolvimento da solução e como o projeto funcionará na sua
totalidade, abordando tanto conceitos de domínio do problema como também os requisitos
funcionais e não funcionais.

O quarto capítulo – \nameref{sec:4-Implementacao} – tem como objetivo apresentar a solução desenvolvida
e descrever detalhes de implementação, assim como explicações sobre as decisões tomadas durante o
desenvolvimento do projeto, possíveis alternativas e uma avaliação geral do sistema.

O quinto capítulo – \nameref{sec:5-Conclusoes} – realiza uma síntese dos resultados alcançados com o
desenvolvimento do projeto, limitações encontradas bem como uma perspetiva de futuras melhorias e considerações 
finais sobre o trabalho realizado no âmbito do Projeto de Estágio.

No final do documento são disponibilizados alguns anexos e conteúdos bibliográficos que suportam o trabalho
desenvolvido e que são informações complementares das secções.
